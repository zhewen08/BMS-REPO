\documentclass{sig-alternate-10pt}


\makeatletter
\let\@copyrightspace\relax
\makeatother

%\usepackage{caption}
\usepackage{floatrow}
% Table float box with bottom caption, box width adjusted to content
%\newfloatcommand{capbtabbox}{table}[][\FBwidth]
\usepackage{blindtext}

%\usepackage{sectsty}
\usepackage{graphicx}
%\usepackage{times}
\usepackage{epsfig}
\usepackage{epstopdf}
\usepackage{subfigure}
\usepackage{url}
\usepackage{cite}%
\usepackage{color}
\usepackage{balance}
\usepackage{multirow}
\usepackage[normalem]{ulem}
\usepackage{fmtcount}

\usepackage{times}
\usepackage{cite}
\usepackage{amsfonts,amssymb,amsmath}
\usepackage{balance}
%\usepackage{verbatim}
%\usepackage{ulem} \normalem
\usepackage{algorithm}
\usepackage{algorithmic,eqparbox,array}
\renewcommand\algorithmiccomment[1]{%
    \hfill\///\ \eqparbox{COMMENT}{#1}%
}
\newcommand\LONGCOMMENT[1]{%
    \hfill\///\ \begin{minipage}[t]{\eqboxwidth{COMMENT}}#1\strut\end{minipage}%
}


%\usepackage{algpseudocode}
%\usepackage{rawfonts}
%\include{amsthm.sty}
% Define new commands
%\usepackage{amsthm}
%\usepackage{fmtcount}
%\usepackage{threeparttable}

\newenvironment{noindlist}
 {\begin{list}{*}
 {\leftmargin=0.35em \itemindent=0em \labelwidth = 0em \labelsep = 0.1em \itemsep = 0em \parsep = 0em \topsep = 0em \listparindent = 0.5in}}
 {\end{list}}

\newenvironment{noinditemize}{
\begin{itemize}
  \setlength{\itemsep}{0pt}
  \setlength{\parskip}{-3pt}
  \setlength{\parsep}{-3pt}
}{\end{itemize}}




\newtheorem{definition}{Definition}
\newtheorem{lemma}{Lemma}
\newtheorem{theorem}[lemma]{Theorem}
\renewcommand{\paragraph}[1]{\textbf{#1: }}
%\newif\ifdebugdoc\debugdocfalse
\newif\ifdebugdoc\debugdoctrue


\ifdebugdoc
%% Writing Mode
\newcommand{\fhl}[1]{\uwave{#1}}
\newcommand{\fyi}[1]{\footnote{\textcolor{blue}{fyi:#1}}}
\newcommand{\remind}[1]{\footnote{\textit{[\uwave{Remind:} #1]}}}
\newcommand{\repl}[2]{\textcolor{red}{#1}\textcolor{blue}{\sout{#2}}}
\newcommand{\add}[1]{\textcolor{red}{#1}}
\newcommand{\del}[1]{\textcolor{blue}{\sout{#1}}}
\newcommand{\outline}[1]{\textbf{\colorbox{yellow}{Outline:}\textcolor{red}{#1.}}}
\newcommand{\old}[1]{\large{\colorbox{blue}{Former: #1. }}}

\else
%%Submission Mode
\newcommand{\fhl}[1]{#1}
\newcommand{\fyi}[1]{}
\newcommand{\remind}[1]{#1}
\newcommand{\repl}[2]{#1}
\newcommand{\add}[1]{#1}
\newcommand{\del}[1]{}
\newcommand{\chunyi}[1]{}
\newcommand{\outline}[1]{}
\fi


\begin{document}
\title{NDNREPO: Transparent Permanent Storage for NDN-Ready Content Objects}



\author{Zhe Wen\\
UCLA Computer Science, Los Angeles, CA 90095}

\maketitle

\begin{sloppypar}



% edit version

%\begin{abstract}
This is the abstract
\end{abstract}
\section{Introduction}

Content object repository (REPO) is a very important application that facilitates NDN wire-format data storage and fetching. On the one hand, the REPO listens to interests under certain set of NDN prefixes, and replies the requested content object if available; on the other hand, the REPO accepts command interests signed by certified requesters and performs insertion/deletion operations as requested.

The previous REPO implementation, i.e. ccnr, leveraged its low level storage on raw file system and lacked support for access control and deletion. Its successor migrated to SQLite database and enabled deletion operation. In both versions, however, processing and matching the interest prefix and selectors to search for content object had been the major, if not the only, headache that compromised the simplicity and elegance of REPO design and implementation.

Compared with both ccnr and its SQLite based successor, our new REPO prototype exploits graph database which provides intuitive and natural support for NDN hierarchical naming search and data storage. Specifically, we base our implementation on Neo4J, a popular graph database that comes with its own graph storage model (in contrast to either the relational model or the key-value pair model) and query language (Cypher) that features graph pattern (node, relation, path, etc) search that natively fits into the NDN naming philosophy.

The only possible drawback of the prototype is that we used the TCP based RESTful Neo4J API set for Python, which might become the bottleneck of query performance. However, it is still good enough to reveal the pros and cons of exploiting a graph database for NDN REPO and can be later improved via migrating to another Neo4J driver that accesses the low level storage directly.

Our prototype is constituted by:
\begin{itemize}
    \item REPO driver. Library that provides API set for insertion/deletion/extraction operations on the REPO.
    \begin{itemize}
        \item insertion. wraps given content as an object and insert it into REPO under specified name; insert given content object into REPO under specified name.
        \item deletion.\footnote{to be decided}
        \item extraction. searches REPO for content object as requested by the interest (prefix + selectors).
    \end{itemize}
    \item REPO server. listens to incoming interests under specified prefixes and replies the requested content object if available.
\end{itemize}

For the first prototype release, our REPO prototype is supposed to provide the following functionality:
\begin{itemize}
    \item REPO API set. support for REPO insertion/deletion/extraction operations via REPO driver API set.
    \item data extraction. listens to and replies interests of content object fetching.\footnote{command interests not supported yet}
    \item data insertion. data insertion protocol \footnote{not available now}
    \item data deletion. data deletion protocol \footnote{not available now}
\end{itemize}

\section{Design}

\subsection{Overview}

NDNREPO is based on graph database, which makes the design simple and elegant. The tree-style NDN hierarchical name naturally fits the graph representation. Therefore it is intuitive and effective to split a given NDN name into a sequence of components as nodes, and connecting them with simple relations that leads to the next component. The actual content object can be attached to the leaf component with another relation.

The root component of the NDN names, can be exploited to optimize searching the graph database. The graph database searches for requested patterns (node, relation, path) through traversing the graph, which could be done much more efficiently if the start node is known ahead. The root component of a NDN name (for now it is ``/ndn'', so we have only \emph{one} name tree in the graph database) serves as this start node perfectly and it is therefore natural to trace the requested content object following the name components in order.

\subsection{Nodes and Relations}

Nodes and relations, both can come with a type and multiple schema free properties, constitute the property graph database. We use a node to represent a name component and a relation to connect 2 name components.

For example, given 6-component name ``/ndn/ucla.edu/melnitz/1451/power/seg0'', we use accordingly 6 ``Component'' type nodes to represent each component with the ``component'' property that stores the component content. Specifically, node (root:Component \{component:``ndn''\}) is store in the graph database for the first (root) component. Similarly, node (comp:Component \{component:``ucla.edu''\}) is for the second component. To store the actual content object under this name, we exploit another ``Segment'' type node. This node is attached to the leaf ``Component'' node and has a ``file'' property whose content is Base64-encoded wired format object data. For instance, node (data:Segment \{file:``<Base64-endcoding-data>''\}) stores the actual content object in wire format.

2 types of relations are used to connect graph nodes. The ``CONTAINS\_COMPONENT'' relation connects one name component to the next name component. One example is the relation connecting component \emph{root} and component \emph{comp} in above example. The ``CONTAINS\_SEGMENT'' relation connects the leaf name component  to the actual content object data node. In above example component \emph{leaf} (leaf:Component \{component:``seg0''\}) and segment \emph{data}, which does not corresponds to any component in the name, are connected with this relation.

Differentiating nodes of components and nodes of content object data facilitates identifying the content of a node. The name/selectors based search is limited in the isolated name tree consisting of mere ``Component'' nodes connected by ``CONTAINS\_COMPONENT'' relations. Once we find a node that fulfills all selector requirements under the specified name prefix, its ``CONTAINS\_SEGMENT'' relation directly leads to the content object data requested. In this design, each node has \emph{at most} one outgoing ``CONTAINS\_SEGMENT'' relation.

\subsection{Content Object Data}

The content object data is stored as the ``data'' property of the ``Segment'' nodes. Ideally the wire format content object is supposed to be stored as binary data in the node. However, we need to use Base64-encoded data for storage due to limit of the Python binding exploited in the prototype. 
%\section{Implementation}
\begin{itemize}
\item
What are challenges for the implementation?
\item
How does it address each challenge in the implementation?
\item
What are the software/hardware platforms for the implementation?
\item
Complexity of the implementation?
\begin{itemize}
\item
e.g., lines of code
\item
Does it work with other existing software/hardware platforms?
\item
If not, is it easy to export it to these platforms?
\end{itemize}
\end{itemize}
%\section{Evaluation}
\begin{itemize}
\item
Goal: show how quantitatively good the solution is
\item
Describe the testing scenarios
What devices used, the supporting environment, etc.
\item
Describe the analytical results
Spell out the assumptions and conditions for the analysis
\item
Explain figures, tables, bar charts, etc.
Tell the readers the percentage improvement, the gains etc. Do not expect the readers to get such numbers by themselves from the figures, etc.
\item
Share the insights why the solution provides better results
\item
For those results worse than the existing solutions, explain why they are so
It is okay to share negative results, as long as they are explained why; provide some justification if possible
\item
Provide a short summary of the performance results
The main items for the readers to take home
\end{itemize}
%\section{Discussion}
\begin{itemize}
\item
This section serves as the storage room for the work
\item
If there are messy issues, state here:
Not in the design section, which may distract the readers from your main idea
\item
If there are straightforward extensions of the solution, state here
\item
If there are unaddressed, but important issues, discuss here:
They are basically the loopholes of the work, argue them here
\item
If there are suggestions/improvements to the current solutions, state here"
These are items that authors do not have time to evaluate and test out
\end{itemize}
%\section{Related Work}
\begin{itemize}
\item
Main point to make: the work is significantly different from all the existing solutions!
Not necessarily better
It is not incremental, which extends the existing ones a little bit
\item
Novelty of the problem: one of the following
formulated a NEW problem in this paper!
identified NEW issues to an existing problem
\item
Novelty of the solution
The idea explored in this paper is completely different from all others in the literature
used new techniques borrowed from other areas or fields
No one has done so, I�m the first one
\item
Novelty of the evaluation
used new analysis/experimental methods that no one has used before
Stay at the level as high as possible: the contribution is major, not minor improvements (no need to comment on the detailed level)
Do not discuss the novelty of each component of the solution, only the main idea of the solution
Component novelty is described in the design section already, not here
\end{itemize}
%\section{Conclusion}
\begin{itemize}
\item
Recap of the problem and the solution
\item
Articulate the importance of the solution:
(1) Is it applicable to other areas or problems?
(2) Does it explore new design principles/philosophies that offer new ways to solve many other problems?
\item
Share insights gained and lessons learned
\begin{itemize}
\item
What are the new positive insights gained?
E.g., certain ideas really work
\item
What are the negative lessons learned?
E.g., complex solutions give only marginal improvement
E.g., certain ideas proposed in the literature do not work at all in the tested scenarios
\end{itemize}
\item
Ongoing/future work (optional)
One or two sentences are enough
Not too much, otherwise, the paper sounds work-in-progress that reviewers can reject easily!
\end{itemize}







%%%%%%%%%%%%%%%%%%%%%%%%%%%
% References %
%%%%%%%%%%%%%%%%%%%%%%%%%%%
\small
\balance
\bibliographystyle{abbrv}
%\bibliographystyle{unsrt} %Entries are not ordered alphabetically, but in the order they are first referenced.
\bibliography{reference-list}
%\bibliography{../bib/diversity,../bib/energy-all, ../bib/diversity }

%\input{appendix.tex}


\end{sloppypar}
\end{document}
